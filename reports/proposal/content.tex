\setlength{\headheight}{15pt}
\lhead{\textbf{ITCS-6166}}
\chead{\textbf{Group \#?}}
\rhead{\textbf{Proposal}}

\vspace{-2pt}

\begin{center}
    \begin{Large}
        \noindent\textbf{Chord-API for Information Sharing}
    \end{Large}
\end{center}

\vspace{-10pt}

\begin{multicols}{2}

    \begin{center}
        \begin{large}
            \noindent\textbf{Team Members}
        \end{large}
    \end{center}
    \begin{center}
        \begin{tabular}{|lc|}
            \hline
            \textbf{Brian Borgia} & 800939320 \\ \hline
            \textbf{Trevon Cornwell} & 801215017 \\ \hline
            \textbf{David Gary} & 801325583 \\ \hline
            \textbf{Lucky Kodwani} & 801276339 \\ \hline
            \textbf{Joseph Mauney} & 801008273 \\ \hline
        \end{tabular}
    \end{center}

    \begin{center}
        \begin{large}
            \noindent\textbf{Primary Sources}
        \end{large}
    \end{center}

    \begin{center}
        \begin{tabular}{|c|}
            \hline
            \href{https://dl.acm.org/doi/10.1145/964723.383071}{Chord}\cite{Chord} \\ \hline
            \href{https://dl.acm.org/doi/10.1145/1830483.1830489}{Ant Colony Systems}\cite{Pheromone} \\ \hline
            \href{https://eclass.teicrete.gr/modules/document/file.php/TP326/%CE%98%CE%B5%CF%89%CF%81%CE%AF%CE%B1%20(Lectures)/Computer_Networking_A_Top-Down_Approach.pdf}{Kurose \& Ross}\cite{KuroseRoss} \\ \hline
            \href{https://dl.acm.org/doi/10.1109/CCGRID.2009.39}{Self-Chord}\cite{Self-Chord} \\ \hline
        \end{tabular}
    \end{center}

\end{multicols}

\begin{large}
    \noindent\textbf{Introduction}
\end{large}

\vspace{5pt}

When considering how to share real-time information across a networked environment, a number of concerns about information staleness must be addressed.
Crude solutions like constant update polling and flooding are not viable in large-scale networks.
This restriction led to the advent of smarter solutions like the Chord protocol, which creates a distributed hash table (DHT) to store and retrieve information in O(log n) time.

\vspace{5pt}

Information sharing is especially important in the context of intelligent agent systems.
As the number of agents in a system increases, so does the requirement for efficient information sharing.
In this project, we aim to build an easy-to-use, scalable, and efficient Chord-API that can be plugged into a variety of applications, but we will explicitly test it on an ant colony inspired system for pathfinding.

\vspace{5pt}

\begin{large}
    \noindent\textbf{Midterm Progress Report Goals}
\end{large}

By the midterm progrss update, our goal is to have the Chord-API fully implemented and tested with a basic CLI interface.
This will allow us to test the API's flow of information and ensure that node failures and joins are handled properly.
During this step, we will also display thorough documentation of the API's usage and how one can customize it to their needs.
The majority of this will be done through YAML configuration files, allowing the user to specify the maximum number of nodes, the distribution scheme for keys, and what type of information is being stored at each node.

\vspace{5pt}

\begin{large}
    \noindent\textbf{Final Demonstration Goals}
\end{large}

At the final demonstration, we will utilize our Chord-API as a backbone for the information sharing necessary in a game where ant agents are tasked with finding the nearest food source and maximizing their colony's food supply.
This will require us to create the ant agent game while ensuring that it is written to maximize the Chord-API's utility.
Our hope is that we will have a live demonstration of the game/Chord-API in action so that the class can participate in the game and see how the Chord-API is used to share information between the agents.

\pagebreak