\setlength{\headheight}{15pt}
\lhead{\textbf{ITCS-6166}}
\chead{\textbf{Group \#?}}
\rhead{\textbf{Proposal}}

\vspace{-2pt}

\begin{center}
    \begin{Large}
        \noindent\textbf{Chord-API for Information Sharing}
    \end{Large}
\end{center}

\vspace{-10pt}

\begin{multicols}{2}

    \begin{center}
        \begin{large}
            \noindent\textbf{Team Members}
        \end{large}
    \end{center}
    \begin{center}
        \begin{tabular}{|lc|}
            \hline
            \textbf{Brian Borgia} & 800939320 \\ \hline
            \textbf{Trevon Cornwell} & 801215017 \\ \hline
            \textbf{David Gary} & 801325583 \\ \hline
            \textbf{Lucky Kodwani} & 801276339 \\ \hline
            \textbf{Joseph Mauney} & 801008273 \\ \hline
        \end{tabular}
    \end{center}

    \begin{center}
        \begin{large}
            \noindent\textbf{Primary Sources}
        \end{large}
    \end{center}

    \begin{center}
        \begin{tabular}{|c|}
            \hline
            \href{https://dl.acm.org/doi/10.1145/964723.383071}{Chord}\cite{Chord} \\ \hline
            \href{https://dl.acm.org/doi/10.1145/1830483.1830489}{Ant Colony Systems}\cite{Pheromone} \\ \hline
            \href{https://eclass.teicrete.gr/modules/document/file.php/TP326/%CE%98%CE%B5%CF%89%CF%81%CE%AF%CE%B1%20(Lectures)/Computer_Networking_A_Top-Down_Approach.pdf}{Kurose \& Ross}\cite{KuroseRoss} \\ \hline
            \href{https://dl.acm.org/doi/10.1109/CCGRID.2009.39}{Self-Chord}\cite{Self-Chord} \\ \hline
        \end{tabular}
    \end{center}

\end{multicols}

\begin{large}
    \noindent\textbf{Introduction}
\end{large}

\vspace{5pt}

\begin{large}
    \noindent\textbf{Methodology}
\end{large}

By the midterm progrss update, our goal is to have the Chord-API fully implemented and tested with a basic CLI interface.
This will allow us to test the API's flow of information and ensure that node failures and joins are handled properly.
During this step, we will also display thorough documentation of the API's usage and how one can customize it to their needs.
The majority of this will be done through YAML configuration files, allowing the user to specify the maximum number of nodes, the distribution scheme for keys, and what type of information is being stored at each node.

\vspace{5pt}

\begin{large}
    \noindent\textbf{Results}
\end{large}

\pagebreak